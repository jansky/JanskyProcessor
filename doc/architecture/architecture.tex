\documentclass[12pt,letterpaper]{article}
\usepackage[latin1]{inputenc}
\usepackage{amsmath}
\usepackage{amsfonts}
\usepackage{amssymb}
\usepackage{graphicx}
\author{Ian Duncan}
\title{JanskyProcessor Architecture}
\begin{document}
\maketitle
\newpage
\tableofcontents
\section*{Introduction}
This document aims to (eventually) be a complete reference to the JanskyProcessor architecture, instruction set, and computer. I know the name is horrible, but otherwise the project would be nameless. Also, please pronounce the name "yahn-skee" (janski in IPA).
\section{Data Storage}
The JanskyProcessor computer has a Von Neumann architecture. Program and data memory are both stored in RAM. There also exists thirteen registers, some of which are used for special purposes such as the stack and instruction pointers.

\subsection{RAM}


\end{document}