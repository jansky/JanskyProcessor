\documentclass[12pt,letterpaper]{article}
\usepackage[latin1]{inputenc}
\usepackage{amsmath}
\usepackage{amsfonts}
\usepackage{amssymb}
\usepackage{graphicx}
\author{Ian Duncan}
\title{JanskyProcessor Architecture}
\begin{document}
\maketitle
\newpage
\tableofcontents
\section*{Introduction}
This document aims to (eventually) be a complete reference to the JanskyProcessor architecture, instruction set, and computer. I know the name is horrible, but otherwise the project would be nameless. Also, please pronounce the name "yahn-skee" (janski in IPA).
\section{Data Storage}
The JanskyProcessor computer has a Von Neumann architecture. Program and data memory are both stored in RAM. There also exists thirteen registers, some of which are used for special purposes such as the stack and instruction pointers.
\newline\newline
\noindent All data is stored in the little-endian format \footnote{The emulator does not currently work on big-endian machines due to this fact.}
\subsection{Data Size}

The JanskyProcessor architecture is a 32-bit architecture. Although it is not required, in practice data is 32-bit aligned in memory, as the \texttt{put} and \texttt{cpy} instructions operate on 32 bit sections of memory. 
\newline\newline
\noindent Below is a table with the words used to refer to different sizes of data.

\begin{center}
\begin{tabular}{ | l | l |}
\hline
\textbf{Name} & \textbf{Size in Bytes} \\ \hline
Byte & 8 \\ \hline
WORD & 16 \\ \hline
DWORD & 32 \\ \hline
\end{tabular}
\end{center}

\subsection{Registers}

Below is a table of all the registers and their properties and uses.

\begin{center}
\begin{tabular}{ | l | l | l | l | p{3cm} |}
\hline
\textbf{Name} & \textbf{Hex ID} & \textbf{Can Read?} & \textbf{Can Write?} & \textbf{Use} \\ \hline
\texttt{ar1} & \texttt{0x00} & Yes & Yes & Integer arithmetic \\ \hline
\texttt{ar2} & \texttt{0x01} & Yes & Yes & Integer arithmetic \\ \hline
\texttt{ar3} & \texttt{0x02} & Yes & Yes & Integer arithmetic \\ \hline
\texttt{ar4} & \texttt{0x03} & Yes & Yes & Integer arithmetic \\ \hline
\texttt{ar5} & \texttt{0x04} & Yes & Yes & Integer arithmetic \\ \hline
\texttt{ip} & \texttt{0x05} & Yes & Only with \texttt{jmp} instructions & Pointer to next instruction \\ \hline
\texttt{bp} & \texttt{0x06} & Yes & Yes & Pointer to instruction to return to (not used currently) \\ \hline

\end{tabular}
\end{center}

\subsection{RAM}

RAM is used to store both program and data memory. The JanskyProcessor architecture uses a 32-bit flat addressing model, limiting the memory size to 4 GB.
\newline\newline
\noindent The first program to be executed is always loaded at address \texttt{0x00000000}.

\subsection{The Stack}

The stack always starts at the highest addressable location in memory (i.e. if RAM is 4096 bytes long, the stack starts at address \texttt{0xFFF}), and continues until the stack base. Each item on the stack is always a DWORD.


\end{document}